\documentclass[12pt]{article}
\usepackage[latin1]{inputenc}
\usepackage[brazilian]{babel}
%%\usepackage{sbc-template}  
\usepackage{amsmath}
\usepackage{graphicx,url}
%\usepackage{graphicx}
\usepackage{color}
\usepackage[num]{abntex2cite}
\usepackage{listings}

\usepackage{float}
     
\sloppy

\begin{document} 
\begin{center}
{\Large \textbf{Universidade de S\~ao Paulo} \\ Instituto de Ci�ncias Matem�ticas e de Computa��o  \\ [0.5cm]}

\end{center}
\begin{center}

\textbf{ Disciplina: Engenharia de Seguran�a (SSC-0747)\\[1.0 cm] }
 
\textmd{Professora Doutora Kalinka Regina Lucas Jaquie Castelo Branco \\[1.0cm] }

\textmd{Aluno: Eduardo Louren�o Pinto Neto, n�USP: \\[0.2cm] }
\textmd{Aluno: Emanuel Carlos de Alc�ntara Valente, n�USP: 7143506\\[0.2cm] }
\textmd{Aluno: Sibelius Seraphini, n�USP: \\[0.2cm] }


\textmd{Data: 26/05/2013\\[4cm] }



\Large \textbf{Trabalho 2 - Criptografia\\ [1cm]}

\end{center}

\pagebreak

%mudar o nome dessa secao (t� varzea!)
\section{Compara��o de algoritmos de criptografia}

Foram comparados 4 algoritmos..

\subsection{Algoritmo 1 - RSA0 (ex)}

\subsection{Algoritmo 1 - RSA1 (ex)}

\subsection{Algoritmo 1 - RSA2 (ex)}

\subsection{Algoritmo 1 - RSA3 (ex)}



\section{Implementa��o dos Algoritmos}

Foram implementados 2 algoritmos ....

\subsection{Algoritmo 1}

\subsection{Algoritmo 2}

\subsection{Compara��o dos 2 algoritmos implementados}








\end{document}


%% <!-- Local IspellDict: brasileiro -->
%% <!-- Local Variables: -->
%% <!-- mode:flyspell -->
%% <!-- End: -->
